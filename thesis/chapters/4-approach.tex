\chapter{Approach}\label{chap:approach}

% Rust has its own share of problems, but a lot of the aforementioned issues will
% be adressed by using Rust in this implementation.


\section{Problem Description}\label{sec:problem}

As noted in \secref{sec:task}, the main goals include testing the integration
of Rust within Python, by implementing a counter-version to the original Python
implementation of the iterative part of the ICE algorithm, and then comparing
it with the original Python-implementation as well as the recent implementation
of the KR-algorithm in C++. It will be tested if the memory efficiacy can be
improved, also how well the parallelization using Rust works really, and how
the integration from Python to Rust works. For C/C++ there exist the Python
headers, and extensive support from Pythons package manager \verb|pip| and
common packaging tools like \verb|setuptools|. For Rust, as it turns out, the
support using Python headers is not as easy, and the support for building packages
from Pythons side is early at best. Both of this will be covered in
\secref{sec:integration} and \secref{sec:api} respectively.


% Wolff: Jede Position im Genom hat in der Summe die gleiche anzahl an
%        interaktionen mit anderen Positionen des Genoms.

\section{Iterative Correction and Eigenvector decomposition (Algorithm)}\label{sec:ICE}

The ICE algorithm was proposed by Imakaev et al. 2012
\cite{imakaev2012iterative}, described in detail in their supplementary
material, and defines an iterative correction approach as follows.

The goal is to obtain the the vector of biases $B_i$ and the true contact map
$T_{ij}$ with their relative contact probabilities. This is done by explicitly
solving the system of the following two equations:

\begin{equation} \label{eq:1}
O_{ij} = B_i B_j T_{ij}
\end{equation}
\begin{equation} \label{eq:2}
\sum^N_{i=1, |i-j|>1} T_{ij} = 1
\end{equation}

% [...] \\
% For a range of distributions, the maximum-likelihood solution for $B_i$ under
% these constraints is the solution of the above equaltion, which can be found
% by the simple iterative procedure described below. After the vector of biasesll
% is computed, the corrected map of relative contact probabilities is obtained
% by $T_{ij} = O_{ij} / (B_i B_j)$.

\eqref{eq:1} is stating, that when applying $B$ back again on our corrected
matrix $T_{ij}$, it will be the same as the original matrix $O_{ij}$ again.
\eqref{eq:2} states, that the sum over the corrected matrix, over arbitrary
elements in the upper left triangle, but only one from each column, sums up to
one. $T_{ij}$ is doubly stochastic ($\forall_j\sum^N_{i=1}T_{ij} = 1$ and
$\forall_i\sum^N_{j=1}T_{ij} = 1$), \extend{does it say that it is the same?
why is this actually valid?}


In the algorithm, this is achieved in the following way. First, $W_{ij}$, a
copy of $O_{ij}$ is created. This matrix will converge to $T_{ij}$ during the
iterative process. The elements of $B$ are initialized with $1$.

% Algorithmically, the iterative correction is implemented as follows. We start
% by creating a working copy of the matrix $O_{ij}$, denoted $W_{ij}$ as the
% iterative process gradually changes this matrix to $T_{ij}$. We initialize
% the iterative procedure by setting each element of the vector of total biases
% $B$ to 1.

\begin{equation}\label{eq:3}
    S_i = \sum_j W_{ij}
\end{equation}
\begin{equation}\label{eq:4}
    \Delta B_i = S_i / mean(S)
\end{equation}
Each iteration starts by first calculating the coverage by summing up each row
(or column, matrix is symmetric so this does not matter) (\eqref{eq:3}) and
additional biases based on this by dividing them through their own mean
(\eqref{eq:4}).

\begin{equation}\label{eq:5}
    W_{ij} = W_{ij} / \Delta B_i \Delta B_j
\end{equation}
\begin{equation}\label{eq:6}
    B_i = B_i \cdot \Delta B_i
\end{equation}
Then $W_{ij}$ is iterated by dividing by $\Delta B_i \cdot \Delta B_j$
(\eqref{eq:5}), after which $B_i$ is iterated by multiplying with the current
biases (\eqref{eq:6}). $W_{ij}$ accumulates divisions by $\Delta B_i$, just as
$B_i$ accumulates the products of $\Delta B_i$. This is repeated until the
variance of $\Delta B$ becomes negegible, at which point $W_{ij}$ has converged
to $T_{ij}$.
% We begin each iteration by calculating the coverage $S_i = \sum_j W_{ij}$.
% Next, additional biases $\Delta B_i$ are calculated by renormalizing $S_i$ to
% have the unit mean $\Delta B_i = S_i /$ mean $(S_i)$. We then divide $W_{ij}$
% by $\Delta B_i \Delta B_j$ for all $(i, j)$ and update the total vector of
% biases by multiplying by the additional biases. Iterations are repeated until
% the variance of the additional biases becomes negligible; at this point
% $W_{ij}$ has converged to $T_{ij}$.






\section{Introducing Rust}\label{sec:Rust}

\subsection{History}\label{sec:RustHistory}

Rust started out 2006 as a personal project of Graydon Hoare, a Mozilla
employee \cite{rustbegin}. The Mozilla foundation started sponsoring in 2009
\cite{rustbegin}. The first compiler was written in OCaml, but 2011 rustc was
able to compile itself with the llvm backend \cite{rustcompile}. Starting with
Rust 1.0, which itself got released on May 15, 2015 \cite{rustversions}, there
was a new stable point version every six weeks \cite{rustversions}. Early on,
Rust had frequent breaking changes \cite{rustchanges}, recently barely anyone
had breakage when updating \cite{rustupdate}.

\subsection{Categorization}\label{sec:RustCategory}

Rust is classified as a high-level language, even though fine low-level control
is possible. This is due the high amount of high-level zero-cost abstractions.
Rust has a type system with strong guarantees, promising e.g. that all
references (pointers) are valid, or thread safety (memory access from other
threads does not result in data races / undeterminism). This is possible
through concepts such as ownership and lifetimes. Even though one can program
in an object oriented way, Rust is primarily not object-oriented. Additionally
it is imperative, procedural, generic and functional.

\subsection{Language Features}\label{sec:RustFeatures}

\paragraph{Syntax:}\label{sec:syntax}
The concrete syntax seems similar to C/C++ (curly braces, function signatures),
however it is more similar to that of ML or Haskell. A particular example for
this case are type classes called traits here, similar to C++ templates but
inspired from Haskell, supporting polymorphism and generic types. Generic
parameters can be constraints, by requiring that generic type to implement a
certain Trait.

\paragraph{Memory safety:}\label{sec:memsafe}
Rust is designed to be memory safe, and does not permit dangling pointers, null
pointers, data races in safe code, or usage of uninitialized variables. In case
a \verb|Null| is needed, the Option-type is provided. Thus, the compiler can
guarantee the validity of all references at compile time using its
borrow-checker.

% The system is designed to be memory safe, and it does not permit null
% pointers, dangling pointers, or data races in safe code.[30][31][32][33] Data
% values can only be initialized through a fixed set of forms, all of which
% require their inputs to be already initialized.[34] To replicate the function
% in other languages of pointers being either valid or NULL, such as in linked
% list or binary tree data structures, the Rust core library provides an option
% type, which can be used to test if a pointer has Some value or None.[31] Rust
% also introduces added syntax to manage lifetimes, and the compiler reasons
% about these through its borrow checker.

\paragraph{Memory management:}\label{sec:memmanage}
Rust does not have a garbage collector, instead, the resource acquisition is
initialization (RAII) convention is used, with optional reference counting.
Resource management is deterministic with very little overhead, favoring stack
allocation without implicit boxing. References are not run time counted, as
their usage is verified at compile time. with this, memory safety can be
guaranteed, limiting possible undefined behaviour tremendously.

% Rust does not use an automated garbage collection system like those used by
% Go, Java, or the .NET Framework. Instead, memory and other resources are
% managed through the resource acquisition is initialization (RAII) convention,
% with optional reference counting. Rust provides deterministic management of
% resources, with very low overhead.[citation needed] Rust also favors stack
% allocation of values and does not perform implicit boxing. There is also a
% concept of references (using the & symbol), which do not involve run-time
% reference counting. The safety of using such pointers is verified at compile
% time by the borrow checker, preventing dangling pointers and other forms of
% undefined behavior.

\paragraph{Ownership:}\label{sec:ownership}
In Rust, all values have a unique owner, and the scope of the value is the same
as the owners. Immutable references can be passed using \verb|&T|, mutable
references by \verb|&mut T|. Pass by value works by passing \verb|T|. Only
\textbf{one} mutable reference can exist at any point, or any number of
immutable ones. This is enforced at compile-time.

\paragraph{Borrowing:}\label{sec:borrowing}
Borrowing results directly from the concept of ownership. As mentioned, only one mutable
borrow (reference) can happen at a time, however that borrowing variable can
further borrow it to other variables or functions. The number of immutable
borrows is unlimited, meaning there can be multiple references reading but not
modifying part of the memory. This is necessary to guarantee memory safety, as
only one mutable reference can write to it at any point in time, wherever that
is (in the code).

\paragraph{Lifetimes:}\label{sec:lifetimes}
Lifetimes are the simple concept of keeping track how long each variable and
each reference is alive, this is preventing the simple case of variables
going out of scope but returning a pointer to it. The compiler can keep track
of this in even much more complex environments.
Non-Lexical-Lifetimes\footnote{Introduced in version 1.31 for the 2018-edition,
and 1.36 for the 2015-edition. Before that, Code Example 2 would not compile.}
also work together with borrowing, resulting in variables returning their
borrow before the end of the scope, as can be see in the first example of
\secref{sec:examples}.

\paragraph{Tooling:}\label{sec:tools}

The reason Rust is loved \cite{rustloved} this much is at least partly due to
tooling. This includes the dedicated package manager \verb|cargo|, the linter
\verb|rustfmt| or \verb|cargo-fix| (a subcommand that can be added later), that
format Rust code after predefined guidelines, or fix most compiler lint warnings
automatically and upgrade to newer conventions, respectively.

More information about Rust can be found
here\footnote{\url{https://www.rust-lang.org/}, accessed 2019-06-26}.




\subsection{Code Examples}\label{sec:examples}

\paragraph{Demonstrating Ownership and Borrowing:}

By executing \coderef{own1} the output from \outref{out1} will be returned.

\code{code_ownership.rs}{own1}

\console{output1.txt}{out1}

As the compiler is complaining, v needs a {\em mutable} borrow to modify
\verb|v|, however \verb|x| still has an {\em immutable} borrow! The borrow from
\verb|x| cannot be ended yet, because it should be printed later. As the mutable
borrow from \verb|v| could modify it in a way such that the reference \verb|x|
would be invalid (e.g. delete \verb|v|), this is a potential memory safety problem.
However it is fine to print x first, and modify \verb|v| afterwards. In
\outref{out2} can be seen what happens when printing the second element of
\verb|v| instead of \verb|x| in the last line, and printing \verb|x| before
adding the second element of \verb|v| (as shown in \coderef{own2}).

\code{code_ownership2.rs}{own2}

\console{output2.txt}{out2}

\paragraph{Demonstrating Ease of Parallelization:}\label{demo:par}

Due to the strong guarantees from the compiler, Memory Safety can be extended
to thread safety. In \coderef{par1} the function \verb|heavy_operation| is
applied to every element in the list. For this, over the list \verb|somelist|
is being iterated, and \verb|heavy_operation| mapped over by taking the values
from the map-closure - Closures are comparable with lambda-functions from
Python, in that they can take arguments and are unnamed functions.

\coderef{par1} and \coderef{par2} demonstrate how easy it is to turn
non-parallel code (\coderef{par1}) in parallelized code (\coderef{par2}).
The difference here being the imported \verb|rayon::prelude::*| and instead of
\verb|iter| now applying \verb|par_iter| to the original list.

\code{code_parallel1.rs}{par1}

\code{code_parallel2.rs}{par2}



\subsection{Advantages of Rust}\label{sec:RustAdvantages}

% Two parts, one, language, second, for this use-case

\paragraph{In General:} As seen in \secref{sec:RustFeatures} and
\secref{sec:examples}, Rust has several high-level features ready to use,
supporting the developer tremendously. Strong compiler guarantees allow easy
parallelization and using Rust libraries without worry, since the compiler will
complain if it used wrong. In combination with semantic
versioning\footnote{\url{https://semver.org/}, accessed 2019-06-26}, this
allows adding and using dependencies without worry. The result are many small
libraries which depend upon each other, instead of few big ones. Dependencies
upon one hundred Rust libraries are not uncommon, and the strong guarantees
from the compiler enforce correct usage.

\paragraph{For this project:} Even though high modularity exists, the Rust
ecosystem is comparatively young. meaning even though many libraries exist,
they are not as complete as their Python/C/C++ counterparts. Since no
implementation of CSRMatrix (compressed sparse row matrix) having the
required features, even though several existed, it was implemented again,
adding the one necessary feature, and not implementing any other. This has
the advantage of being a very specific solution, possibly faster and smaller
than the general ones available.

Also, as seen in \todo{Link section results}, Parallelization can decrease
the elapsed computation time.

% The advantage of using Rust over using \verb|numpy| / \verb|scipy| from Python
% for this work might not be immediately obvious, since the CSRMatrix had to be
% implemented. We considered using the \verb|numpy| C-API for a while, but its
% too big to be actually useful for our use-case.

% The advantage here is actually much simpler: since we barely need any of the
% features provided, implementing them ourselves is not much work and gives us
% way more fine grained control as to what is actually happening.

% This includes the parallelization of some parts of the code, which might not
% have been possible if ther were some other library doing things in its own way
% (numpy using the C-API would be an example here).

% A future advantage is also the modularity of Rust code, meaning in the future
% additional external libraries (and with them, features) could easily be integrated.




\subsection{Disadvantages of Rust}\label{sec:RustDisadvantages}

% Two parts again, one, language, second, for this use-case

\paragraph{In General:} General disadvantages of Rust include the young ecosystem
with slightly less diversity, or too much feature-incomplete diversity. The
steep learning curve in the beginning, needs to be mentioned, since Rusts
features cannot be selectively activated. Compared to languages like GO, Rust
has considerably higher initial compile times. Even though breakage rarely
happens \cite{rustupdate}, a new point-release happens every six weeks,
frequently introducing new features requiring time to fully understand. The
userbase is still growing, and many features frequently used in other
languages, such as \verb|async| or specializations, are not yet available for
users of the stable compiler.

\paragraph{For this project:} In particular, the unavailability of a CSRMatrix
implementation with the needed features is concerning. Thus, the current
implementation does not have any more features than are needed for the current
algorithm, being only a tiny subset of the features provided by the
\verb|scipy| implementation.



\subsection{Comparing Rust and Python}\label{sec:rustvspython}

Rust and Python are two quite different programming languages, a direct
translation is not possible. Both implementations are the same semantically,
however details differ. Since Rust has a much finer control of memory and the
applying of functions to data structures, some operations have been explicitly
separated while others have been combined.

Depending on the questions asked, either language may prevail. While Python
allows (seemingly) faster development cycles, it is more prone to runtime
errors and library misuse. For Rust, the compiler provides strong guarantees,
requiring more development time up front but less to fix bugs.



\subsection{Comparing Rust with C/C++}\label{sec:rustvscc++}

\begin{table}[ht]
    \ra{1.3}
\begin{tabular}{@{}lllr@{}}
    \textbf{Speed comparison} & C & Rust & C++ \\ \midrule
    n-body & 7.49 & \textbf{5.72} & 8.18 \\
    binary-trees & 3.48 & \textbf{3.15} & 3.79 \\
    pidigits & \textbf{1.75} & \textbf{1.75} & 1.89 \\
    reverse-complement & 1.78 & 1.61 & \textbf{1.55} \\
    spectral-norm & 1.98 & \textbf{1.97} & 1.98 \\
    fannkuch-redux & \textbf{8.61} & 10.23 & 10.08 \\
    k-nucleotide & 5.01 & 5.25 & \textbf{3.76} \\
    fasta & 1.36 & 1.47 & \textbf{1.33} \\
    mandelbrot & 1.65 & 1.96 & \textbf{1.5} \\
    regex-redux & \textbf{1.46} & 2.43 & 1.82 \\ \midrule
    Fastest in: & 3/10 & 4/10 & 4/10 \\
\end{tabular}
    \caption[Runtime Comparison Between Rust/C/C++]{\textbf{Comparing Runtime speeds.} Runtime measured in seconds. Numbers from the benchmarksgame\footnotemark.}
    \label{tab:runtime}
\end{table}

\footnotetext{
Rust comparison with C: \url{https://benchmarksgame-team.pages.debian.net/benchmarksgame/fastest/rust.html},
and with C++: \url{https://benchmarksgame-team.pages.debian.net/benchmarksgame/fastest/rust-gpp.html},
both accessed 2019-06-26 }

As can be seen in \tabref{tab:runtime}, the runtime of Rust is very close to
that of C and C++ (as is memory, not compared here). Both C and C++ are
currently much more widely used, but there are already voices calling to
replace C++ with
Rust\footnote{\url{https://hub.packtpub.com/will-rust-replace-c/}, accessed
2019-06-26}.


Resource needs may be close to the same, but from a developer standpoint Rust
has consistently been the `most loved Language` \cite{rustloved} for the last
four years, whereas both C and C++ both rank considerably high in the category
`dreaded` \cite{rustloved}. Reasons why Rust may be such a loved language are
listed in \secref{sec:RustAdvantages}. Due to the barely integrated external
static analysis done for C/C++-code, both C and C++ are more prone to memory
bugs \cite{pronememory}, thus having higher developer time requirenments.





\subsection{Choosing the right API to call Rust from Python}\label{sec:api}

There are three main ways to execute Rust code from Python. In the following,
the feasability of them for this thesis is evaluated. And compared in
\tabref{apis}.

\begin{stretchpars}
\paragraph{rust-cpython:} One common way is rust-cpython. This library requires
Rust 1.25 or higher, the current stable version at the time of writing is 1.33.
Rust-cpython grants access to the Python gil (global interpreter lock) with
which Python code can be evaluated and Python objects modified. The resulting
library can easily be
\end{stretchpars}

\code{code_cpython.rs}{cpython}

imported into Python after renaming the compiled library. Native Rust code
requires some wrapping first, as can be seen in \coderef{cpython}. The code is
taken from the \verb|rust-cpython| example which can be found
here\footnote{\url{https://github.com/dgrunwald/rust-cpython}, accessed
2019-06-26}. \todo{correctly align!!}

This wrapping, though quite common and based on the Python C-API makes
it hard to write purely idiomatic Code in Rust. Since Python is directly
affected, the interactions with Python need to be considered while writing
Rust-Code, including the affecting of memory Python is managing. This was
deemed too much complexity overhead.


\paragraph{pyO3:} Another common approach is using the \verb|pyO3|-library, which
started off as a fork of rust-cpython, but has since seen drastic changes.
However, as \verb|pyO3| continues to use unstable Rust features, it is only
possible to compile this library with the nightly version of the compiler. Even
though most unstable features have been stabilizied by now,
Specialization\footnote{\url{https://github.com/rust-lang/rust/issues/31844},
accessed 2019-06-26} is still missing several steps, as it is not sound
regarding the type system yet, and not all questions are answered. Nightly
features are subject to tremendous change, making this option a questionable at
best for this implementation, as the goal is to have a stable implementation.


\paragraph{Generate dylib:} The described way in the official rust docs is to
create a \verb|dylib| and import the resulting library in the respective
language
dynamically\footnote{\url{https://doc.rust-lang.org/1.2.0/book/rust-inside-other-languages.html}, \\
accessed 2019-06-26}.  Here is no renaming necessary, but the
communication between Rust and Python is more low-level. The main wrapper
is on the side of Python, transforming Arguments to pointers and
C-Representations. Rust needs to export an interface usable from C, which can
be done by a simple \verb|extern|. Rust has the \verb!\#[no_mangle]! and
\verb!\#[repr(C)]! (procedural) macros, preventing the compiler to mangle
(renaming of functions) and guaranteeing the representation in the memory
layout to be as it would be in C, when they need to be exported.


\begin{table}[ht]
    \ra{1.3}
\begin{tabular}{@{}lllr@{}}
    \textbf{API Comparison} & rust-cpython & pyO3 & dylib \\
    \midrule
    any renaming needed             & Yes    & Yes & \textbf{No} \\
    stable Rust                     & \textbf{Yes}    & No  & \textbf{Yes} \\
    platform-specific compilation flags & Yes & Yes & \textbf{No} \\
    using Memory managed by Python  & Yes    & Yes & \textbf{Optional} \\
    additional implementation effort                 & Medium & Medium & \textbf{Low} \\
    difficulty of creating python packages        & \textbf{Easy}\footnotemark  & \textbf{Easy}\footnotemark[\value{footnote}] & Normal \\
    \midrule
    Good in: & 2/6 & 1/6 & \textbf{5/6} \\
\end{tabular}
    \caption[Comparison of different Interfaces between Rust and Python]{\textbf{Comparing different API interfaces.}}
    \label{tab:interfaces}
\end{table}

\footnotetext{This was not tested, but pyo3-pack is a zero-configuration
package builder, \url{https://github.com/PyO3/pyo3-pack}, accessed 2019-06-26.}

When comparing the available options as can be seen in \tabref{tab:interfaces},
dylib appers to be the best option. See \secref{sec:approach} for details
regarding the communication between Python and Rust. \todo{link exact paragraph}


\subsection{Integration of Rust in Python}\label{sec:integration}


The integration of both C and C++ in Python are considerably straightforward
and well-supported. Even though Rust is considerably new, a multitude of option
exists, especially for both \verb|rust-cpython| and \verb|pyO3|. However, none
fully offered what was needed for packaging Python with a dylib, and building
it.
\verb|setuptool-rust|\footnote{\url{https://github.com/PyO3/setuptools-rust},
accessed 2019-06-26} sounded promising, but requires \verb|rust-cpython| or
\verb|pyO3| bindings to work correctly. A different \verb|setuptools| extension
called \verb|milksnake|\footnote{\url{https://github.com/getsentry/milksnake},
accessed 2019-06-26}, a package specifically for the distribution of
dynamically linked libraries with Python, promised to be capable of including
both our library and the python-wrapper for it. This promise however turned out
to be undocumented.

The implementation now uses \verb|milksnake| to resolve depencies like the
not-yet compiled library, but the setuptools mechanisms for including the
library itself and the Python wrapper.




% \newpage
\subsection{Using this Implementation}\label{sec:using}

\subsubsection{Installation}\label{sec:install}

\todo{for installing using conda add dependencies}

\verb|smb| can be run on any Unix-based operating system (tested using ubuntu-18.04)
with Conda, Python and common development packages, e.g.
\verb!libopenssl-dev!, \verb!python3-dev!, \verb!build-essential! and the like
installed. For the installation itself just enter \verb|conda install -c kargf smb|.

\textbf{For using, not building, installation of Rust is not needed.}


\subsubsection{Build}\label{sec:build}

Detailed build instructions can be found in the corresponding GitHub
repository: \url{https://github.com/fkarg/HiC-rs}. For this, an installation of
Rust, as well as \verb|conda| and \verb|pip| are needed.


\section{General Approach}\label{sec:approach}


\subsection{Beginning}

In the beginning, after consulting related material
(\cite{imakaev2012iterative}, \cite{lieberman2009comprehensive} and
\cite{wingett2015hicup}), the Python-implementation was studied for details.
Then, the feasability of communicating between Rust and Python was tested
first, see \secref{sec:api} for the selection of the API. For this, small
examples passing a simple list back and forth were implemented.

\subsection{Feasability Testing}

Next, the general feasability of the project was tested, this includes looking
for usable libraries. The Python-implementation showed a strong dependency for
\verb|numpy| and \verb|scipy|, but nothing similar was available for Rust.
There are several good linear algebra libraries, and ports of numpy, however no
CSRMatrix implementation providing usable iteration over its rows. There
existed at least three different CSRMatrix implementations at the time of
research, however as all of them were considerably lacking in features compared
to the \verb|scipy| implementation and not many features were needed, it was
implemented seperately with only the needed features, including being
constructed through a C-API interface, and no effort was made to add the needed
feature to any of the other implementations. During the implementation, tests
for the CSRMatrix written in Rust were written in Python, testing the
integration as well. This cought bugs like interpreting data as a different
data type.


\subsection{Implementation of the Algorithm}

Initially, the writing of the algorithm happened by trying to translate the
algorithm line by line from Python. As this is not possible due to the
considerably different programming languages, it happened paragraph-wise,
rather. This first transcription was considerably naive, and not idiomatic for
Rust. Even though a \verb|numpy| port existed, it was not used as the actually
needed operations were considerably little, and probably faster without porting
to \verb|numpy| first. Most operations themselves were on \verb|scipy|
CSRMatrix implementation, and no port of \verb|scipy| was available. This way,
direct control over all critical parts of execution was assumed.


\subsection{Testing and Bugfixing}

Just as it does in Haskell (and probably any other strongly typed language),
the concept of compiler driven development exstist. This means that e.g. for
refactoring, some part is modified, and then compiler errors and warnings are
fixed one after another. As soon as the compiler does not complain any more,
the new functionality might not yet be implemented, but the new field or
renamed function is now being referenced correctly everywhere. A test suite in
Rust as well as in Python was set up. In rust, variables are immutable by
default, testing led to changing a borrowed reference to a borrowed mutable
reference at two points, and looking into slices again in detail.


\subsection{Idiomatic Rust and summing high fractions}

\code{code_infsum1.rs}{inf1}

\code{code_infsum2.rs}{inf2}

While gradually transforming the naive Python-translation to idiomatic Rust, at
some point results ended up being \verb|NaN| pretty fast. Two step sbefore
results ended up being \verb|NaN| the situation was as described in
\coderef{inf1}. Here everything worked fine, summing up over a considerably
short list produced the expected results. Only one iteration later these
elements have been multiplied with several factors, their product being
\verb|0.16|. This in and by itself should not have changed anything big,
however looking at \coderef{inf2} their sums were now not exactly what one
would expect at first, with one non-zero number among zeros being equal to
another one and a sum of zeros being \verb|inf| for some reason. As it turns
out, the factor they have been subject to was indeed \verb|0.16|, however it
was \verb|0.16| with high fraction values. This means that summation of
\verb|0.16| and \verb|0.0| (the zero also having a high fraction) is being
sufficiently inaccurate to not be accurately represented by floating point
values, resulting in \verb|0.4| instead of the expected \verb|0.16| after
adding only four times. The same happened with the summation of the
innocious-looking \verb|0.0|. They had high fractions from the original
multiplication by \verb|0.16000000000000003|, their continued summation
probably resulting in an overflow. This new number just happens to be one of
the many representations of \verb|inf|.


\subsection{Packaging}


Part of the goal is to package the code for easy use from the HiCExplorer as a
\verb|conda| package. The only Python dependency after \verb|setuptools| and
\verb|pip| ended up being \verb|milksnake|, an extension for \verb|setuptools|
for packaging dynamically linked libraries, particularly ones written in Rust.
This is described in \secref{sec:integration} in more detail. As
\verb|milksnake| was not available as a conda package, it ended up getting
ported. Initially, \verb|conda skeleton pypi milksnake| created a package that
even conda could not build, with the issue being that milksnake was only
provided as a \verb|*.zip| file and conda had hardcoded the format
\verb|*.tar.gz|. After doing this, a travis build server (with tests) was set
up, to continuously monitor further progress.


\subsection{Parallelizing}

As is demonstrated in \secref{demo:par}, the parallelization of Rust code can
be really easy. This is why this was done last, as nothing really needed to
change. by any tremendous amount. There is another variant for parallelizing
code in Rust (with explicit threads) which was also tested, but it provided no
further benefit and proved to be too much overhead. This might not be the case
for truly big matrices however.

\todo{Add summary regarding Rust}
\draft{put note somewhere that this server is different from the earlier
mentioned travis test server, make distinction clear}

