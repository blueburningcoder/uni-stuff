\chapter{Related Work}\label{chap:relatedwork}

The main focus of this work is the implementation of the ICE matrix correction
algorithm in Rust, as well as the testing of this Rust implementations
integration with Python. Related work includes the original implementation of
the ICE algorithm in Python as well as the recent implementation of the
similar KR-algorithm in C++. Disadvantages and advantages of the respective
implementations will be evaluated in the following.

A formulaic description of the implemented algorithm can be found in
\secref{sec:ICE}.


\section{Python implementation}\label{sec:python}

The original implementation was written in Python, since HiCExplorer is
written in Python. This implementation used common python dependencies
extensively, including the compressed sparse row matrix (CSRMatrix)
implementation from \verb|scipy|, as well as the scientific number manipulation
library \verb|numpy|.

\subsection{Advantages}

The implementation itself is considerably short, the file having only 86 Lines,
part of which are imports and frequent comments. The advantage of using Python
here is verbosity, as most lines are not for the functionality itself, but for
timing, logging and ease of debugging. With the iteration of the ICE algorithm
starting no earlier than line 40, most are high-level \verb|numpy| /
\verb|scipy| commands, some being themselves implemented in C/C++ to be
sufficiently fast.

Another advantage of Python in general is fast implementation times, which are
possible through the concise syntax making the spotting of mistakes easier.
\subsection{Disadvantages}

The downsides of this implementation are that data structures in Python are
extensively objectified, meaning they require more working memory. Additionally,
even though Python has existing parallelism, a global interpreter lock prevents
multiple threads to use the same parts of code and memory without duplication
(Python is only interpreted usually, but this still holds for compiling with
CPython). Since Python already has comparatively high memory requirements,
it is not practicable to add the same amount for every further core (details
about memory needs can be found in \secref{sec:memory}).


For further reference, the Python-implementation can be found
here\footnote{\url{https://github.com/deeptools/HiCExplorer/blob/master/hicexplorer/iterativeCorrection.py},
accessed 2019-06-26}.



\section{KR-Algorithm}\label{sec:KR}

What follows is a short description to the Knight-Ruiz-algorithm from
\cite{knight2013fast}. Essentially, the algorithm is using conjugate gradients
to converge faster, but those steps are computationally more expensive
compared to steps from the ICE algorithm.

\subsection{Implementation}

The KR-algorithm was originally implemented in Matlab, but here we compare with
a version implemented in C++. Calls from Python to C++ can be done over the
C-API with considerable help through Python-header files. There is extensive
support for building and packaging C/C++ code for Python available.

\subsection{Advantages}

A commonly mentioned advantage of C++ is the speed of execution, and
fine-grained control over Memory available. Implementations in C++ can be
several orders of magnitude faster than their respective implementation in
Python. An advantage of the Algorithm itself is that as long as the matrix
itself has total support, it will converge (meaning that at least one diagonal
has only positive nonzero values, which can be artificially done setting zeros
to some small positive value). Thus, it will converge for many more matrices
than the ICE-algorithm.

\subsection{Disadvantages}

Even though the execution is fast, the development process tends to be slow.
This is due to the free memory control, which is hard to get right as this
requires the upkeep of several implicit assumptions at several places. As these
assumptions are implicit only, it is easy to forget them or 'cut corners' when
not possible. Those bugs leading to segmentation-faults, i.e. accessing invalid
memory, are notoriously hard to find, as they do not follow deterministic logic.
Parallelism is even harder to add, since data races and race conditions, both
non-deterministic, and other sources for hard-to-get right problems are added.
Additionally, the syntax is considerably complex, making it rather hard to
understand.

For further reference, the implementation of the KR algorithm can be found
here\footnote{\url{https://github.com/deeptools/Knight-Ruiz-Matrix-balancing-algorithm},
\\ accessed 2019-06-26}.




