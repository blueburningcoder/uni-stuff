\chapter{Related Work}\label{chap:relatedwork}

\extend{some section here to fill another page}

Since the main work is the implementation in Rust as well as the testing of an
integrating in Python, related work includes the original implementation in
Python as well as the recent implementation of the KR-algorithm in C++.
Disadvantages and advantages of the respective implementations will be
evaluated in the following.

The description of the implemented algorithm can be found in \secref{sec:ICE}.


\section{Python implementation}\label{sec:python}

The original implementation was written in Python, since HiCExplorer is
written in Python. This implementation is using common python dependencies
extensively, including the Compressed Sparse Row Matrix (CSR Matrix)
implementation from \verb|scipy|, as well as the scientific number manipulation
library \verb|numpy|.

\subsection{Advantages}

The implementation itself is considerably short, the file having only 86 Lines,
including imports and frequent comments. The advantage of using Python here is
showing, as most lines are not for functionality itself, but for timing,
logging and ease of debugging. With the iteration itself starting no earlier
than Line 40, most are High-Level \verb|numpy| / \verb|scipy| commands, some
being themselves implemented in C/C++ to be sufficiently fast.

Another advantage of Python in general are fast implementation times, which are
possible through the concise syntax making the spotting of mistakes easier.

\subsection{Disadvantages}

\draft{Include: Multithreading is locking the execution of the same part of
code, processes allow that but require the same memory for writing in it,
reading is fine}

The downsides of this implementation being that datastructures in Python are
extensively objectified, meaning they require more working memory, and that
even though Python has existing parallelism, a global interpreter lock (python
is only interpreted usually, but this still holds for compiling with cpython)
prevents multiple threads to use the same parts of memory without duplication.
Since Python already has comparatively high memory requirements \draft{link
high memory needs}, it is not practicable to add the same amount for every
further core.

For reference, the Python-implementation can be found
here\footnote{\url{https://github.com/deeptools/HiCExplorer/blob/master/hicexplorer/iterativeCorrection.py},
accessed 2019-06-26}.



\section{KR-Algorithm}\label{sec:KR}

What follows is a short description to the Algorithm known as Knight-Ruiz from
\cite{knight2013fast}. Essentially, the algorithm is computing the same things,
however it is taking advantage of conjugate gradients to solve the iterative
steps faster.

\subsection{Implementation}

The KR-algorithm was originally implemented in Matlab, here we compare with a
version in C++. Calls from Python to C++ can be done over the C-API with
considerable help through Python-header files.

\subsection{Avdantages}

A commonly mentioned advantage of C++ is the speed of execution, and
fine-grained control over Memory available. However, implementations in C++ can
be several orders of magnitude faster than their respective implementation in
Python. An advantage of the Algorithm itself is that as long as the matrix
itself has total support (meaning that at least one diagonal has only positive
nonzero values, this can be artificially done setting zeros to some small
positive value), it will converge. Thus, it will converge for way more matrices
than the ICE-algorithm.

\subsection{Disadvantages}

Even though the execution is fast, the development process tends to be slow.
This is due to the free memory control, which is hard to get right as this
requires upholding of implicit assumptions at several places. As these
assumptions are implicit only, it is easy to forget them or 'cut corners' when
not possible. Those bugs leading to Segmentation-faults (accessing invalid
memory) are notoriously hard to find, as they do not follow determinism.
Parallelism is even harder to add, since data races (also nondeterministic) and
other sources for hard-to-get right problems are added. Additionally, the
syntax is considerably complex, making it rather hard to understand.
For reference, the implementation of the KR algorithm can be found
here\footnote{\url{https://github.com/deeptools/Knight-Ruiz-Matrix-balancing-algorithm},
\\ accessed 2019-06-26}.

