\chapter{Related Work}\label{chap:relatedwork}

% Related Work:
\todo{introduce original implementation in python and KR-algorithm in C++}

\section{Deeptools}\label{sec:deeptools}


\todo{introduce the whole deeptools framework}


\section{HiCExplorer}\label{sec:hicexplorer}
\todo{introduce hicexplorer}
% analysing Hi-C data: building this interaction-matrix, correcting it,
% analysing it, compute stuff with it, ..., visualising the results.

\section{Python implementation}\label{sec:python}

As the deeptools-framework is mainly written in Python, and as this algorithm
is an important one, it was originally implemented in Python. This
implementation however requires too much working memory, limiting the size of
usable matrices, which is why a different implementation was asked for.

Rust and Python are two quite different programming languages, which is why a
direct ``translation'' is not even possible. Both are semantically speaking the
same, but details differ. Since Rust has a much finer control about memory and
the applying of functions to data structures, some operations have been
separated while others have been combined.

The biggest difference, however, is that in Rust certain tasks can easily be
parallelized - even after originally writing it for only one core. For smaller
matrices the overhead can be quite big, for one of our test matrices the optimum was using 'only' three of the available 32 cores (this still gave an improvement of \todo{get numbers!}).

% but for bigger matrices this is qutie the improvement.

\section{KR-algorithm}\label{sec:KR}

\todo{read paper + introduce here}

\todo{is that really everything related?}
