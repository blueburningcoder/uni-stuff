\chapter{Introduction}\label{chap:introduction}

% \todo{include feedback!}
% Feedback:
% Why do we care about the 3D structure -> about the structure, influencing
% enhancer-promoter interactions, existing in reality in 3D and we have barely
% information, we want to understand this
%
% gene regulation also happening in 3D, way bigger why we want this
%
% around 1.5 Pages here
%
% removing the Eigenvector part, only mention it at the analysis, or when
% introducing the algorithm
%
% also mention: we have that data -> why do we want to correct it to begin with
%
% 'storytelling' introduction:
% - first: HiC and why it is interesting
% - errors that can happen to the data
% - super small introduction to rust
% - problem introduction
%
% sequencing errors:
% - some reagions are more sensible for biotin labeling enrichments
% - PCR duplicates: regions have multiple reads
% - mapping: certain locations are unclear, could be mapped to several different ones
% - random interactions -> may be technical errors or correct
% - mapping: location wrong, highly repetitive regions
% - illumina: more likely to get certain mapping errors
% - mapping algorithm has bias itself
% - mismatches likelier for longer reads
% => Errors do exist, but do not really know which ones or how to filter them out


Enhancers are small protein-binding regions of DNA, usually searched for within
a distance of 1 Mbp up or downstream of promoters
\cite{pennacchio2013enhancers}, which initiate the process of transcription.
Structures such as DNA loops (see \figref{fig:HiCMatricesModels}), however, can
bring together an enhancer with a promoter over distances far greater than 1
Mbp \cite{pennacchio2013enhancers} due to the complex three-dimensional
structure of the chromosome. Using Hi-C technology, interactions over the whole
genome \cite{lieberman2009comprehensive} and even across different genomes can
be measured \cite{rao20143d}. Topologically-associating domains (TADs) describe
regions which preferentially interact with one another over others.
Computational methods using Hi-C data can identify hundreds of thousands of
putative enhancers and their target genes by searching for these TADs
\cite{ron2017promoter}.

In this work, data obtained from Hi-C will be used (see \chapref{chap:results}
for details about the data). Hi-C is a method for acquiring the 3D information
of genomes. This is done by strapping together parts of the genome that are
close by, and cutting the genome apart with restriction enzymes, combining the
ends of strapped-together fragments, and using high throughput methods to
sequence them. This is explained in detail in \secref{sec:HiC}.

Such technologies tend to suffer from unwanted biological and technical
factors, such as different chromatin states \cite{pmid19693276} and
sequencing/mapping errors \cite{pmid21646344} respectively, making them
inherently inaccurate. Biases are unavoidable, and some regions are more
sensitive to biotin labelling enrichment where they will be measured more often
when compared to others (See \secref{sec:pulldown} for relevant use-cases). PCR
artifacts may be one of the reasons \cite{wingett2015hicup}, but multiply
mapped loci also contribute to the ambiguity of the data, introducing even more
potential causes for bias. Specific sequencing methods also have certain
inherent biases \cite{aird2011analyzing}. The measured Hi-C interactions are
often questionable, since it is unclear if some of these interaction are
spatially nearby points, or if they are artefacts of technical errors, or even
just randomly-occurring interactions.

However, a basic but strong assumption about the structure of the genome can be
made, which is that every location has the same amount of interactions with
other locations, as every other location. The data does not always show this
due to the several aforementioned unwanted biological and technical factors.
Corrective algorithms such as ICE \cite{imakaev2012iterative} (Iterative
Correction and Eigenvector decomposition, \secref{sec:ICE}) or KR
\cite{knight2013fast} (Knight-Ruiz, \secref{sec:KR}) can however still be
applied to correct the matrix.


\newpage
\section{Task Definition}\label{sec:task}


In the three-dimensional space of a cell the DNA forms a complex structure that
is distributed all over the place. Due to this structure, there exist many
points of contact in the DNA which form noticeable sub-structures, such as DNA
loops. Many of these measured points of contact are random interactions, or
measurement errors that need to be corrected. For this task, a Python
implementation exists but it is limited for high resolution data due to high
memory usage. This thesis aims to re-implement a more resource efficient method
in Rust.

The main goals include: testing the integration between Rust, a systems
programming language recently gaining in popularity (details in
\secref{sec:Rust}), and Python; testing the inter-process communication between
these two languages; and comparing the two current matrix correction
implementations, ICE in Python and KR in C++. More information about these can
be found in \secref{sec:python} and \secref{sec:KR}, respectively. The overall
goal is to try to implement a more resource efficient version, able to make
effective use of parallel computing.

